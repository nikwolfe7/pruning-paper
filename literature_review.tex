\section{Related Work}
Neural network over-fitting is fundamentally a problem arising from the use of too many free parameters. Regardless of the number of weights used in a given network, as \cite{segee1991fault} assert, the representation of a learned function approximation is almost never evenly distributed over the hidden units, and the removal of any single hidden unit at random can actually result in a total network fault. \cite{mozer1989using} suggest that only a subset of the hidden units in a neural network actually latch on to the invariant or generalizing properties of the training inputs, and the rest learn to either mutually cancel each other out or begin over-fitting to the noise in the data. Determining which elements are unnecessary and removing them outright is therefore a well-founded approach to improving network generalization, and simultaneously provides a way to reduce their size in memory. 


The generalization performance of neural networks has been well studied, and apart from pruning algorithms many heuristics have been used to avoid overfitting, such as dropout (\cite{srivastava2014dropout}), maxout (\cite{goodfellow2013maxout}), and cascade correlation (\cite{fahlman1989cascade}), among others. However, these algorithms do not explicitly prioritize the reduction of network memory footprint as a part of their optimization criteria per se, (although in the opinion of the authors, Fahlman's cascade correlation architecture holds great promise in this regard.) Computer memory size and processing capabilities have improved so much since the introduction of pruning algorithms in the late 1980s that space complexity has become a relatively negligible concern. The proliferation of cloud-based computing services has furthermore enabled mobile and embedded devices to leverage the power of massive data and computing centres remotely. In this domain, however, it is also reasonable to suggest that certain performance-critical applications running on low-resource devices could benefit from the ability to use neural networks locally. 

At present there are few (if any) mechanisms specifically designed to shrink neural networks down in order to meet an externally imposed constraint on byte-size in memory. Without explicitly removing parameters from the network, one could use weight quantization to reduce the number of bytes used to represent each weight parameter, as investigated by \cite{balzer1991weight}, \cite{dundar1994effects}, and \cite{hoehfeld1992learning}. Of course, this method can only reduce the size of the network by a factor proportional to the byte-size reduction of each weight parameter. 

Another method which has recently gained popularity is using the singular values of a trained weight matrix as basis vectors from which to derive a compressed hidden layer. 

* describe SVD stuff

If we wanted to continually shrink a network to its absolute minimal size in an optimal manner, we might accomplish this using any number of off-the-shelf pruning algorithms, such as Skeletonization (\cite{mozer1989skeletonization}), Optimal Brain Damage (\cite{lecun1989optimal}), or later variants such as Optimal Brain Surgeon (\cite{hassibi1993second}). In fact, we borrow much of our inspiration from these antecedent algorithms, with one major variation. 




The aforementioned strategies all focus on the targeting and removal of \textit{weight} parameters. Scoring and ranking individual weight parameters in a large network computationally expensive, and generally speaking the removal of a single weight from a large network is a drop in the bucket in terms of reducing a network's core memory footprint. We therefore train our sights on the ranking and removal of entire neurons along with their associated weight parameters. We argue that this is more efficient computationally as well as practically in terms of quickly reaching a target reduction in memory size. Our approach also attacks the angle of giving downstream applications a realistic expectation of the minimal increase in error resulting from the removal of a specified percentage of neurons from a trained network. Such trade-offs are unavoidable, but performance impacts can be limited if a principled approach is used to find candidate neurons for removal. 

* brief overview of approach